%Copyright 2014 Jean-Philippe Eisenbarth
%This program is free software: you can 
%redistribute it and/or modify it under the terms of the GNU General Public 
%License as published by the Free Software Foundation, either version 3 of the 
%License, or (at your option) any later version.
%This program is distributed in the hope that it will be useful,but WITHOUT ANY 
%WARRANTY; without even the implied warranty of MERCHANTABILITY or FITNESS FOR A 
%PARTICULAR PURPOSE. See the GNU General Public License for more details.
%You should have received a copy of the GNU General Public License along with 
%this program.  If not, see <http://www.gnu.org/licenses/>.

%Based on the code of Yiannis Lazarides
%http://tex.stackexchange.com/questions/42602/software-requirements-specification-with-latex
%http://tex.stackexchange.com/users/963/yiannis-lazarides
%Also based on the template of Karl E. Wiegers
%http://www.se.rit.edu/~emad/teaching/slides/srs_template_sep14.pdf
%http://karlwiegers.com
\documentclass{scrreprt}
\usepackage{listings}
\usepackage{underscore}
\usepackage[bookmarks=true]{hyperref}
\usepackage[utf8]{inputenc}
\usepackage[english]{babel}
\usepackage{xcolor,colortbl}
\hypersetup{
    bookmarks=false,    % show bookmarks bar?
    pdftitle={Software Requirement Specification},    % title
    pdfauthor={Jean-Philippe Eisenbarth},                     % author
    pdfsubject={TeX and LaTeX},                        % subject of the document
    pdfkeywords={TeX, LaTeX, graphics, images}, % list of keywords
    colorlinks=true,       % false: boxed links; true: colored links
    linkcolor=blue,       % color of internal links
    citecolor=black,       % color of links to bibliography
    filecolor=black,        % color of file links
    urlcolor=purple,        % color of external links
    linktoc=page            % only page is linked
}%
\def\myversion{0.1 }
\definecolor{LightCyan}{rgb}{0.88,1,1}
\date{}
%\title
\usepackage{hyperref}
\begin{document}

\begin{flushright}
    \rule{16cm}{5pt}\vskip1cm
    \begin{bfseries}
        \Huge{SOFTWARE REQUIREMENTS\\ SPECIFICATION}\\
        \vspace{1.9cm}
        for\\
        \vspace{1.9cm}
        App FH++\\
        \vspace{1.9cm}
        \LARGE{Version \myversion}\\
        \vspace{1.9cm}
        Prepared by \\ {\small Mohamad Arastu, Raphael Arzberger, Martin Fuhrmann}
        \vspace{1.9cm} \\
        \today\\
    \end{bfseries}
\end{flushright}

\tableofcontents


\chapter*{Revision History}

\begin{center}
    \begin{tabular}{|c|c|c|c|}
        \hline
	    Date & Reason For Changes & Version\\
        \hline
	    14.09.2022 & project start & 0.1\\
        \hline
    \end{tabular}
\end{center}

\chapter{Introduction}

\section{Purpose}
Die App FH++, welche im Rahmen dieses Software-Projektes entwickelt wird, soll für die Studierenden der FH Campus Wien ein nützliches Tool darstellen,
welches als eine sinnvolle Ergänzung zu den bereits bestehenden Diensten (FH Portal und Moodle) genutzt werden kann.
Das Ziel ist den Usern der App eine Möglichkeit zu bieten das Studium möglichst effizient zu gestalten.
Durch die Nutzung der Schnittstellen des FH Portals und jene des Open-Data-Portals der Stadt Wien, welches Zugriff auf Daten der Wiener Linien ermöglicht,
soll dem User eine effiziente Planung des Studienalltags, ermöglicht werden.
Die Anfahrt zur FH kann durch Kombination von Informationen aus dem individuellen Stundenplan sowie aktuellen Verkehrsdaten des öffentlichen Verkehrs 
optimal gestaltet werden. Wird für die Anfahrt im Normalfall die U1 benützt, so soll die App bei Störungen im U-Bahn-Netz den User vor einer potenziell längeren
Anfahrtszeit warnen.
Für die Navigation innerhalb des Gebäudes der FH, soll der User eine Navigationshilfe mittels Augmented Reality nutzen können.   
Außerdem soll es die Möglichkeit geben Emails des FH Mail Accounts abrufen und versenden zu können.

\section{Organisatorische Einbettung}
Das Projekt findet im Rahmen des Unterrichtes der ILV Software-Engineering statt. Nutzen an der Applikation sollten die Studierenden der Fachhochschule Campus Wien haben.

\chapter{Funktionale Anforderungen}

\section{Use Case Diagramm}

\section{Use Case Descriptions/ Produktkriterien}
\subsection{Allgemeine Muss-Kriterien}
\subsection{Allgemeine Soll-Kriterien}
\subsection{Allgemeine Kann-Kriterien}
\subsection{Use cases}

\begin{center}
\begin{tabular}{|c|c|c|}
	\hline
	\rowcolor{LightCyan}
	\rule[-1ex]{0pt}{2.5ex} Use Case & EIN Beispiel \\
	\hline
	\rule[-1ex]{0pt}{2.5ex} Actors &  \\
	\hline
	\rule[-1ex]{0pt}{2.5ex} Description &  \\
	\hline
	\rule[-1ex]{0pt}{2.5ex} Stimulus &  \\
	\hline
	\rule[-1ex]{0pt}{2.5ex} Response &  \\
	\hline
	\rule[-1ex]{0pt}{2.5ex} Kriterien &  \\
	\hline
	\rule[-1ex]{0pt}{2.5ex} Comments &  \\
	\hline
\end{tabular}
\end{center}

\chapter{Nicht Funktionale Anforderungen}
\section{zb Anforderungen an die DatenhaltungDatenhaltung}

\end{document}
